% \usetheme[compress, ricet, numbers]{Rice}
\usetheme[compress,numbers]{Ggcm}

% preamble for GGCM study-formatted latex/beamer slides

\definecolor{darkestblue}{rgb}{.024,.149,.239} % darkest blue 04263E 0.024 0.149 0.239
\definecolor{darkblue}{rgb}{.047,.243,.388} % darker than official 073D64 0.047 0.243 0.388
\definecolor{blue}{rgb}{.063,.306,.486} % ggcm official blue 094D7E 0.063 0.306 0.486
\definecolor{lightblue}{rgb}{.071,.337,.337} % lighter than official 0A558B 0.071 0.337 0.537
\definecolor{lightestblue}{rgb}{.106,.494,.780} % lighest blue 0E7CCA 0.106 0.494 0.780
\definecolor{gray}{rgb}{.608,.612,.627} % ggcm official gray  9D9CA0


\definecolor{blu}{rgb}{.0,.0,.50}

% note: beamer output size: 128mm x 96mm (5.04in x 3.78in)

% \setbeamercolor{frametitle}{fg=gray,bg=blue}
% \usecolortheme{dolphin}
% \setbeamercolor*{palette primary}{use=structure,fg=gray,bg=blue}
% \setbeamercolor*{palette secondary}{use=structure,fg=gray,bg=blue}
% \setbeamercolor*{palette tertiary}{use=structure,fg=gray,bg=blue}
% \setbeamercolor*{palette quaternary}{fg=white,bg=black}

\newcommand{\vb}{\vspace{\baselineskip}}
\newcommand{\vh}{\vspace{.5\baselineskip}}
\newcommand{\vq}{\vspace{.25\baselineskip}}
\newcommand{\vm}{\vspace{-.5\baselineskip}}

\newenvironment<>{noteblock}[1]{%
\setbeamercolor*{block title}{bg=lightestblue,fg=gray}% These two lines added
\setbeamercolor*{block body}{bg=lightestblue!60}% to the "block" env. definition
\begin{actionenv}#2%
\def\insertblocktitle{#1}%
\par%
\usebeamertemplate{block begin}}
{\par%
\usebeamertemplate{block end}%
\end{actionenv}}

\newenvironment<>{myboldblock}[1]{%
\setbeamercolor*{block title}{bg=gray!20,fg=black}% These two lines added
\setbeamercolor*{block body}{bg=gray!5}% to the "block" env. definition
\begin{actionenv}#2%
\def\insertblocktitle{\footnotesize{\textbf{#1}}}%
\par%
\usebeamertemplate{block begin}}
{\par%
\usebeamertemplate{block end}%
\end{actionenv}}

\newenvironment<>{mylistingblock}[1]{%
\setbeamercolor*{block title}{bg=blue!10,fg=black}% These two lines added
\setbeamercolor*{block body}{bg=gray!2}% to the "block" env. definition
\begin{actionenv}#2%
\def\insertblocktitle{\footnotesize{\textbf{#1}}}%
\par%
\usebeamertemplate{block begin}}
{\par%
\usebeamertemplate{block end}%
\end{actionenv}}

\newenvironment<>{bluenote}[1]{%
\setbeamercolor*{block title}{bg=blue,fg=white}% These two lines added
\setbeamercolor*{block body}{bg=blue!20}% to the "block" env. definition
\begin{actionenv}#2%
\def\insertblocktitle{#1}%
\par%
\usebeamertemplate{block begin}}
{\par%
\usebeamertemplate{block end}%
\end{actionenv}}

\newenvironment<>{mydefinition}[1]{%
\setbeamercolor*{block title}{bg=blue!50,fg=black}% These two lines added
\setbeamercolor*{block body}{bg=blue!20}% to the "block" env. definition
\begin{actionenv}#2%
\def\insertblocktitle{#1}%
\par%
\usebeamertemplate{block begin}}
{\par%
\usebeamertemplate{block end}%
\end{actionenv}}

\setlength{\labelsep}{1.0em}
\newenvironment{mydescription}[1]
  {\begin{list}{}%
   {\renewcommand\makelabel[1]{\color{blu}{##1}\hfill}%
   \settowidth\labelwidth{\makelabel{#1}}%
   \setlength\leftmargin{\labelwidth}
   \addtolength\leftmargin{\labelsep}}}
  {\end{list}}

\setlength{\labelsep}{1.0em}
\newenvironment{bethdescription}[1]
  {\begin{list}{}%
   {\renewcommand\makelabel[1]{\color{blu}\hfill{##1}}%
   \settowidth\labelwidth{\makelabel{#1}}%
   \setlength\leftmargin{\labelwidth}
   \addtolength\leftmargin{\labelsep}}}
  {\end{list}}

\newenvironment{myindentpar}[1]%
{\begin{list}{}%
         {\setlength{\leftmargin}{#1}}%
         \item[]%
}
{\end{list}}

% \setbeamercolor{postit}{fg=blue,bg=gray}
% \setbeamercolor{block body}{bg=gray!10}
% \setbeamercolor{block title}{bg=gray!20,fg=black}

\renewcommand{\thefootnote}{\fnsymbol{footnote}}

% package below is to fix the error "No room for a new \dimen" which showed up in August of 2014
\usepackage{etex}
\usepackage{siunitx}
\usepackage{booktabs}
%\usepackage{ctable}
\setlength{\heavyrulewidth}{0.1 em}
\newcommand {\otoprule}{\midrule[\heavyrulewidth]}
\usepackage{ulem}
\usepackage{listings}
\usepackage[absolute,overlay,showboxes]{textpos}
% line below is commented out to avoid grid display when using
% the package handoutWithNotes
%\usepackage[colorgrid,texcoord]{eso-pic}
\usepackage{tikz}
\usetikzlibrary{shapes.callouts}
\usepackage{multirow}
\usepackage{colortbl}

\setlength{\jot}{10pt}

%adjust the TPHorizModule and TPHorizModule units to the displayed mm %grid
\TPGrid{128}{96}
\TPMargin{5pt}
\textblockrulecolour{red}
\TPshowboxesfalse
%\textblockcolour{gray!20}

%puts a graphic at the absolute position described by the grid
%#1 x, #2 y, #3 width, #4 height, #5 graphic
\newcommand\putpic[5]{%
        \begin{textblock}{#3}(#1,#2)
  \includegraphics[width=#3\TPHorizModule,
  height=#4\TPVertModule]{#5}
     \end{textblock}
}

\newcommand{\code}[1]{\texttt{#1}}

\setbeamertemplate{navigation symbols}{}

% the following 2 lines
% fix the Corrupted NFSS tables problem
% and produce a document with no font related warnings
\usepackage{lmodern}
\usepackage[T1]{fontenc}
%\usepackage[noae]{Sweave}

